\documentclass[a4paper]{article}

\usepackage[a3paper,landscape]{geometry}
\usepackage[francais]{babel}
\usepackage{t1enc}
\usepackage[utf8]{inputenc}
\usepackage{fullpage}
\usepackage{multido}
\usepackage{graphicx}
\usepackage{colortbl}
\definecolor{light-gray}{gray}{0.90}
\arrayrulecolor{light-gray}
% \arrayrulewidth{.5pt}

\usepackage{enumitem}

\setlength{\textwidth}{39cm}
\setlength{\oddsidemargin}{-1.5cm} 
\setlength{\topmargin}{-15mm} 
\setlength{\textheight}{27.8cm}
\renewcommand{\arraystretch}{0}
\setlength{\tabcolsep}{0cm}

\usepackage{pict2e}

\newcommand{\comment}[1]{}

\begin{document}


\thispagestyle{empty}

\setlength{\unitlength}{1cm}

%% Change the geometry for the cups HERE

\begin{picture}(39,8.8)
  \multiput(6,4.8)(3.85,0){8}{
    \circle{7.7}
  }
\end{picture}

\begin{minipage}{6cm}
\vspace{-0.6cm}
  \begin{center}
    {\Huge\sc Cargo-Bot}
  \end{center}
\vspace{-0.4cm}

\subsection*{Présentation et but du jeu}
\vspace{-0.2cm}

Ce jeu consiste \`a programmer une grue (jouée par un participant) qui
manipule des gobelets de couleurs
diff\'erentes.

Pour cela, vous
placerez des
  instructions (\includegraphics[height=\baselineskip]{left} 
\includegraphics[height=\baselineskip]{right} 
\includegraphics[height=\baselineskip]{down} 
\includegraphics[height=\baselineskip]{f0} 
\includegraphics[height=\baselineskip]{f1}
\includegraphics[height=\baselineskip]{f2} 
\includegraphics[height=\baselineskip]{f3}
\includegraphics[height=\baselineskip]{rewind}
\reflectbox{\includegraphics[height=\baselineskip]{rewind}}
) dans la zone de programme ci-contre.

Chaque problème décrit une position de
départ des gobelets et de la grue, ainsi qu'un but à atteindre.

\textbf{Dès que les gobelets sont dans
  cette position, c'est gagné !}

\vspace{-0.2cm}


\subsection*{Déplacements}
\vspace{-0.2cm}

\includegraphics[height=\baselineskip]{left} et 
\includegraphics[height=\baselineskip]{right} d\'eplacent la grue
% à droite ou à gauche 
\emph{à condition de ne pas sortir du terrain de jeu}.
%
%La flèche 

\includegraphics[height=\baselineskip]{down} peut avoir 3
effets différents :

\begin{itemize}[leftmargin=5mm]
\item Si la pince est vide et au-dessus d'un gobelet, elle
  le saisit et remonte.
\item Si la pince contient un gobelet, elle le d\'epose et remonte
  \`a vide.
\item Si la pince est vide et qu'il n'y a rien en dessous, rien ne se passe.
\end{itemize}


\includegraphics[height=\baselineskip]{rewind} et 
\reflectbox{\includegraphics[height=\baselineskip]{rewind}} retournent le gobelet présent dans la pince si elle en contient un.
\end{minipage}
\hfill
%
%% Change program geometry HERE
%
\begin{tabular}{
c@{~}
*{10}{c@{
\begin{tabular}[b]{c}
\includegraphics[width=20mm]{graycond}\\
\includegraphics[width=20mm]{gray}
\end{tabular}
%\hspace{1mm}
}}
}
\includegraphics[width=20mm]{f0}
&&&&&&&&&&\\[5mm]
\includegraphics[width=20mm]{f1}
&&&&&&&&&&\\[5mm]
\includegraphics[width=20mm]{f2}
&&&&&&&&&&\\[5mm]
\includegraphics[width=20mm]{f3}
&&&&&&&&&&\\[5mm]
\end{tabular}
\hfill
\begin{minipage}{6cm}
\vspace{-2mm}

\subsection*{Programmes}

\vspace{-0.2cm}

Chaque programme (F0, F1, F2, F3) est constitué de 10 instructions au
maximum.
Il n'y a pas de \og retour à la ligne \fg{} automatique : une fois
qu'un programme est fini, il ne se passe plus rien !



Les instructions \includegraphics[height=\baselineskip]{f0},
  \includegraphics[height=\baselineskip]{f1}... permettent
d'\emph{appeler} un programme : on quitte (temporairement) le programme
en cours, on exécute celui désigné par cette instruction, puis on
revient au programme de départ s'il n'était pas fini.

\vspace{-0.2cm}

\subsection*{\'Etiquettes
  \includegraphics[height=.5\baselineskip]{red}
 \includegraphics[height=.5\baselineskip]{yellow} 
\includegraphics[height=.5\baselineskip]{green}
 \includegraphics[height=.5\baselineskip]{blue}
}
\vspace{-0.2cm}

%Elles ne sont disponibles que dans les problèmes plus avancés.
On les
place au-dessus d'une instruction, et cette instruction ne s'effectue
que si la pince contient un gobelet de la bonne couleur.
% ( \includegraphics[width =.5cm]{red}, \includegraphics[width
%    =.5cm]{yellow}, \includegraphics[width =.5cm]{green},
%  ou \includegraphics[width =.5cm]{blue}).

L'\'etiquette
  {\it toutes les couleurs} \includegraphics[height=.5\baselineskip]{any} effectue une instruction seulement si
  la pince contient un gobelet%
%(ind\'ependemment de sa couleur)
, et
  l'\'etiquette {\it vide} \includegraphics[height=.5\baselineskip]{none} n'ex\'ecute l'instruction que si la pince est
  vide.


\vspace{3mm}
\includegraphics[height=10mm]{logo_IREM_UGA_transitoire}
\hfill
\includegraphics[height=10mm]{logo-universite-grenoble-alpes}
%\vspace{2mm}

\includegraphics[height=10mm]{inria}%NRIA_CORPO_CMJN}
\hfill
\includegraphics[height=10mm]{logo_MPLSblanc}
\end{minipage}

\vfill

%% Change available instructions HERE

\begin{tabular}{
*{4}{|c@{\includegraphics[width=20mm]{left}}}
*{4}{|c@{\includegraphics[width=20mm]{right}}}
*{3}{|c@{\includegraphics[width=20mm]{down}}}
*{2}{|c@{\includegraphics[width=20mm]{rewind}}}
*{2}{|c@{\reflectbox{\includegraphics[width=20mm]{rewind}}}}
*{1}{|c@{\includegraphics[width=20mm]{f0}}}
*{1}{|c@{\includegraphics[width=20mm]{f1}}}
*{1}{|c@{\includegraphics[width=20mm]{f2}}}
*{1}{|c@{\includegraphics[width=20mm]{f3}}}
}
\hline
&&&&&&&&&&&&&&&&&&\\
\hline
&&&&&&&&&&&&&&&&&&\\[-.3mm]
\end{tabular}

\begin{tabular}{
*{3}{|c@{\includegraphics[width=20mm]{yellow}}}
*{3}{|c@{\includegraphics[width=20mm]{green}}}
*{3}{|c@{\includegraphics[width=20mm]{red}}}
*{3}{|c@{\includegraphics[width=20mm]{blue}}}
*{3}{|c@{\includegraphics[width=20mm]{none}}}
*{4}{|c@{\includegraphics[width=20mm]{any}}}
}
\hline
&&&&&&&&&&&&&&&&&&\\
\hline
\end{tabular}


\end{document}
